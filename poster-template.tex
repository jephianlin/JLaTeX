\documentclass[a0paper, 25pt, portrait]{tikzposter}

\usetheme{Wave}

\usepackage{amsmath, amssymb}
\usepackage{hyperref}
\usepackage{qrcode}

\newcommand{\trans}{^\top}
\newcommand{\adj}{^{\rm adj}}
\newcommand{\cof}{^{\rm cof}}
\newcommand{\inp}[2]{\left\langle#1,#2\right\rangle}
\newcommand{\dunion}{\mathbin{\dot\cup}}
\newcommand{\bzero}{\mathbf{0}}
\newcommand{\bone}{\mathbf{1}}
\newcommand{\ba}{\mathbf{a}}
\newcommand{\bb}{\mathbf{b}}
\newcommand{\bc}{\mathbf{c}}
\newcommand{\bd}{\mathbf{d}}
\newcommand{\be}{\mathbf{e}}
\newcommand{\bh}{\mathbf{h}}
\newcommand{\bp}{\mathbf{p}}
\newcommand{\bq}{\mathbf{q}}
\newcommand{\bx}{\mathbf{x}}
\newcommand{\by}{\mathbf{y}}
\newcommand{\bz}{\mathbf{z}}
\newcommand{\bu}{\mathbf{u}}
\newcommand{\bv}{\mathbf{v}}
\newcommand{\bw}{\mathbf{w}}
\newcommand{\tr}{\operatorname{tr}}
\newcommand{\nul}{\operatorname{null}}
\newcommand{\rank}{\operatorname{rank}}
%\newcommand{\ker}{\operatorname{ker}}
\newcommand{\range}{\operatorname{range}}
\newcommand{\Col}{\operatorname{Col}}
\newcommand{\Row}{\operatorname{Row}}
\newcommand{\spec}{\operatorname{spec}}
\newcommand{\vspan}{\operatorname{span}}


\title{Coefficients of the characteristic polynomial}
\author{Jephian C.-H. Lin}
\date{\today}
\institute{National Sun Yat-sen University}


\begin{document}
\maketitle

\begin{columns}
\column{0.5}
\block{Overivew of the characteristic polynomial}{
The characteristic polynomial of $A$ is defined by $p_A(x) = \det(A -
xI)$.  Usually, we need to expand every term and rearrange it into a
clean form.  For example, 

\[
    \begin{aligned}
      \det(A - xI) &= 
      \det\begin{bmatrix} 
      1 - x & 2 & 3 \\
      4 & 5 - x & 6 \\
      7 & 8 & 9 - x
      \end{bmatrix} \\
      &= (1 - x)(5 - x)(9 - x) + 2\cdot 6\cdot 7 + 3\cdot 4\cdot 8  \\
      &\mathrel{\phantom{=}}- 3\cdot (5 - x)\cdot 7 - 2\cdot 4\cdot (9-x) - (1 - x)\cdot 6\cdot 8 \\
      &= -x^3 + 15x^2 + 18x.
    \end{aligned}
\]

The following formula helps to find the coefficients directly.  The
characteristic polynomial of $A$ can be written as 
\[
    p_A(x) = \det(A - xI) = (-x)^n + s_1(-x)^{n-1} + \cdots + s_n,
\]
where  
\[
    s_k = \sum_{\substack{ \alpha\subseteq[n] \\ |\alpha|=k }}\det(A[\alpha]).
\]  
}

\column{0.5}
\block{An example}{
Let  
\[
    A = \begin{bmatrix} 
    1 & 2 & 3 \\
    4 & 5 & 6 \\
    7 & 8 & 9
    \end{bmatrix}.
\]
We have to go through all subsets $\alpha$ of $[3] = \{1,2,3\}$ of a given size.  

\bigskip

\begin{center}
\begin{tabular}{c|c|ccc|ccc|c}
 & $k = 0$ & \multicolumn{3}{|c|}{$k = 1$} & 
\multicolumn{3}{|c|}{$k = 2$} & $k = 3$ \\
\hline
$\alpha$ & $\emptyset$ & $\{1\}$ & $\{2\}$ & $\{3\}$ &
$\{1,2\}$ & $\{1,3\}$ & $\{2,3\}$ & $\{1,2,3\}$ \\
$A[\alpha]$ & $\begin{bmatrix}~\end{bmatrix}$ &
{\color{red}$\begin{bmatrix}1\end{bmatrix}$} & 
{\color{red}$\begin{bmatrix}5\end{bmatrix}$} &
{\color{red}$\begin{bmatrix}9\end{bmatrix}$} &
{\color{teal}$\begin{bmatrix}1&2\\4&5\end{bmatrix}$} & 
{\color{teal}$\begin{bmatrix}1&3\\7&9\end{bmatrix}$} & 
{\color{teal}$\begin{bmatrix}5&6\\8&9\end{bmatrix}$} & 
            $\begin{bmatrix} 
              1 & 2 & 3 \\
              4 & 5 & 6 \\
              7 & 8 & 9
            \end{bmatrix}$ \\
$\det$ & $1$ & {\color{red}$1$} & {\color{red}$5$} & {\color{red}$9$} &
{\color{teal}$-3$} & {\color{teal}$-12$} & {\color{teal}$-3$} & $0$ \\
\hline
 & $s_0 = 1$ & \multicolumn{3}{|c|}{$s_1 = 15$} & 
\multicolumn{3}{|c|}{$s_2 = -18$} & $s_3 = 0$
\end{tabular}
\end{center}

\bigskip

Thus,  
\[
    \begin{aligned}
      p_A(x) &= (-x)^n + s_1(-x)^{n-1} + \cdots + s_n,
      \\
      &= -x^3 + 15x^2 + 18x.
    \end{aligned}
\]
}
\end{columns}

\begin{columns}
\column{0.5}
\block{Some theory behind the scene}{
When we compute the characteristic polynomial, each column of $A - xI$
can be written as the sum of two vectors, one contains constants while
the other contains $-x$ and $0$'s.  By expanding $\det(A - xI)$ using
the distributive law on each column, we got $2^n$ terms.
\[
    \begin{aligned}
    &\mathrel{\phantom{=}}\det\begin{bmatrix}
     1 - x & 2 & 3 \\
     4 & 5 - x & 6 \\
     7 & 8 & 9 - x
    \end{bmatrix} \\
     &=
    \det\begin{bmatrix}
     -x & 0 & 0 \\
     0 & -x & 0 \\
     0 & 0 & -x
    \end{bmatrix} 
     + 
    \det\begin{bmatrix}
     -x & 0 & 3 \\
     0 & -x & 6 \\
     0 & 0 & {\color{red}9}
    \end{bmatrix} 
     +
    \det\begin{bmatrix}
     -x & 2 & 0 \\
     0 & {\color{red}5} & 0 \\
     0 & 8 & -x
    \end{bmatrix} 
     +
    \det\begin{bmatrix}
     {\color{red}1} & 0 & 0 \\
     4 & -x & 0 \\
     7 & 0 & -x
    \end{bmatrix} 
     + \\
     &\mathrel{\phantom{=}}
    \det\begin{bmatrix}
     -x & 2 & 3 \\
     0 & {\color{teal}5} & {\color{teal}6} \\
     0 & {\color{teal}8} & {\color{teal}9} 
    \end{bmatrix} 
    +
    \det\begin{bmatrix}
     {\color{teal}1} & 0 & {\color{teal}3} \\
     4 & -x & 6 \\
     {\color{teal}7} & 0 & {\color{teal}9}
    \end{bmatrix} 
     + 
    \det\begin{bmatrix}
     {\color{teal}1} & {\color{teal}2} & 0 \\
     {\color{teal}4} & {\color{teal}5} & 0 \\
     7 & 8 & -x
    \end{bmatrix} 
     + 
    \det\begin{bmatrix}
     1 & 2 & 3 \\
     4 & 5 & 6 \\
     7 & 8 & 9
    \end{bmatrix} \\
    &= (-x)^3 + ({\color{red}9 + 5 + 1})(-x)^2 + ({\color{teal}-3 -12 -3})(-x) + 0. 
    \end{aligned}
\]
}

\column{0.5}
\block{My own example}{
I wrote some SageMath code to compute the characteristic polynomial
using this method.  Visit the website below to see more examples.

\begin{tikzfigure}
\label{fig:sageqr}
\qrcode[height=3in]{https://sagecell.sagemath.org/?q=zziwzi}
\end{tikzfigure}

\begin{center}
\href{https://sagecell.sagemath.org/?q=zziwzi}{\tt https://sagecell.sagemath.org/?q=zziwzi}
\end{center}

}
\end{columns}

\end{document}
